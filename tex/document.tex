\documentclass{article}

\usepackage[utf8]{inputenc}  %% 1
%%\usepackage[T2A]{fontenc}      %% 2
\usepackage[russian]{babel}
\usepackage{graphicx}
\begin{document}
	\section{Введение}
	На сегодняшний день очень сложно представить мир без фотографий, к сожалению, очень часто они получаются недостаточно качественными, размытыми. Но нам хотелось бы разглядывать, анализировать фотографии лучшего качества, однако не всегда их можно сделать  ввиду разных обстоятельств.\\
	Для улучшения качества изображений было придумано множество решений, начинающиеся от каких-то простых преобразований до сложных итерационных алгоритмов. Каждый алгоритм имеет свои преимущества и недостатки. 
	\section{Модель размытия}
	Для того, чтобы восстанавливать изображения, для начала надо понять, а каким образом происходит размытие. \\
	Для начала введем обозаначения:\\
	$f(x, y)$ - исходное изображение\\
	$g(x, y)$ - размытое изображение\\
	$h(x, y)$ - искажающая функция(ядро размытия)\\
	$n(x, y)$ - шум\\
	Математически размытие описывается следующим образом:\\
	\begin{center}
	$g(x, y) = f(x, y) \otimes h(x, y) + n(x, y)$\\
	\end{center}
	Где $ f(x ,y) \otimes h(x, y)$ - операция свертки, описывающаяся следующим образом:
	\begin{center}
		$f(x, y) \otimes h(x, y) = \sum\limits_{i=-a}^{a}\sum\limits_{j=-b}^{b} h(-i, -j)f(x+i, y+j) $
	\end{center}
	где $a=m/2, ~ b=n/2$, а $m$ и $n$ - высота и ширина матрицы $h(x, y)$ 
	\section{Постановка задачи}
	Вычислить функцию, обратную свертке весьма нетривиальная задача, для решения которой бьются много людей и разработано множество алгоритмов, не знаю что писать
	\section{Описание методов}
	\subsection{Методы основанные на частотных преобразованиях}
	Линейные методы основаны на теореме о свертке
	\begin{center}
		$g(x, y) = f(x, y) \otimes h(x, y) \Leftrightarrow G(u, v)=F(u, v)*H(u, v)$
	\end{center}
	Где:\\
	$G(u, v)$ - фурье-образ размытого изображения,\\
	$F(u, v)$ - фурье-образ исходного изображения,\\
	$H(u, v)$ - фурье-образ PSF,\\
	$"*"$ - поэлементное умножение\\
	\subsubsection{Инверсный фильтр}
	Благодаря теореме о свертке мы можем выразить $F(u, v)$:
	\begin{center}
		$\^{F}(u, v)=\frac{G(u, v)}{H(um v)}$
	\end{center}
	НУЖНЫ ПРИМЕРЫ!!!\\
	Минус этого метода в том, что если мы не знаем какой у нас шум, то и нормально восстановить мы не сможем
	\subsubsection{Фильтр Винера}
	Также этот метод называется фильтрацией методом минимизации среднеквадратического отклонения.\\
	Фильтр Винера - метод, соединяющий в себе учет свойств искажающей функции и статистических свойств шума в процессе восстановления. Метод основан на рассмотрении изображений и шума как случайных процессов, и задача ставится следующим образом: \textit{ найти такую оценку  $f как поставить крышечку $, для неискаженного изображения $f$, чтобы среднеквадратичное отклонение этих величин было минимальной.}\\
	Стандартное отклонение задается следующей формулой:
	\begin{center}
		$e^2=E\left[ (f- f)^2 \right]  $
	\end{center}
	Предполагается, что выполнениы следующие условия: 
	\begin{enumerate}
		\item Шум и неискаженное изобрадение некоррелированы между собой.
		\item Либо шум, либо нескаженное изображение имеют нулевое среднее значение.
		\item Оценка линейено зависит от искаженного изображения
	\end{enumerate}
	При выполнении этих условий минимум среднеквадратического отклонения достигается на функции, которая задается в частотной облати выражением:
	\begin{center}
		$F(u, v)=\left( \frac{H'(u, v)S_f(u, v)}{S_f(u, v) \left| H(u, v)\right|^2 +S_\eta(u,v) } \right) G(u,v)=\left( \frac{\left| H(u, v) \right|^2}{H(u, v) \left| H(u, v)\right|^2 +S_\eta(u,v) / S_f(u,v)} \right) G(u,v)$
	\end{center}|\
	Где:\\
	$H(u, v)$ - частотное представление искажающей функции.\\
	$H'(u, v)$ - комплексно-сопряженное искажающей функции\\
	$left| H(u, v) right|^2=H'(u,v)H(u, v)$\\
	$S_\eta(u, v) =\left| N(u ,v) \right|^2$ - энергетический спектр шума.\\
	$S_f(u, v)=\left| F(u, v) \right |^2$ - энергитический спектр неискаженного изображения.\\
	$G(u, v)$ - энергетический спектр искаженного изображения\\ \\
	Последнее равенство имеет место из-за того, что произведение комплексного числа на комплексно-сопряженное равно квадрату модуля этого числа.\\
	Этот результат, полученный Винером, известен как оптимальная фильтрация по Винеру.\\
	Когда спектр шума и неискженного изображения неизвестны и не могут быть оценены, часто пользуются следующей формулой:\\
		\begin{center}
		$F(u, v)=\left( \frac{\left| H(u, v) \right|^2}{H(u, v) \left| H(u, v)\right|^2 +K} \right) G(u,v)$
	\end{center}
	Где $K$ - заданная константа.
	КАРТИНКИ!!! МНОГО КАРТИНОК!!!
	\subsubsection{Регулеризация по Тихонову}
	\end{document}